\documentclass[]{rptuseminar}

% Specify that the source file has UTF8 encoding
\usepackage[utf8]{inputenc}
% Set up the document font; font encoding (here T1) has to fit the used font.
\usepackage[T1]{fontenc}
\usepackage{lmodern}

% Load language spec
\usepackage[american]{babel}
% German article --> ngerman (n for »neue deutsche Rechtschreibung«)
% British English --> english

% Ffor bibliography and \cite
\usepackage{cite}

% AMS extensions for math typesetting
\usepackage[intlimits]{mathtools}
\usepackage{amssymb}
% ... there are many more ...


% Load \todo command for notes
\usepackage{todonotes}
% Sebastian's favorite command for large inline todonotes
% Caveat: does not work well with \listoftodos
\newcommand\todoin[2][]{\todo[inline, caption={2do}, #1]{
		\begin{minipage}{\linewidth-1em}\noindent\relax#2\end{minipage}}}

% Load \includegraphics command for including pictures (pdf or png highly recommended)
\usepackage{graphicx}

% Typeset source/pseudo code
\usepackage{listings}

% Load TikZ library for creating graphics
% Using the PGF/TikZ manual and/or tex.stackexchange.com is highly adviced.
\usepackage{tikz}
% Load tikz libraries needed below (see the manual for a full list)
\usetikzlibrary{automata,positioning}

% Load \url command for easier hyperlinks without special link text
\usepackage{url}

% Load support for links in pdfs
\usepackage{hyperref}
\setlength{\footnotesep}{10pt}

% Defines default styling for code listings
\definecolor{gray_ulisses}{gray}{0.55}
\definecolor{green_ulises}{rgb}{0.2,0.75,0}
\lstset{%
  columns=flexible,
  keepspaces=true,
  tabsize=2,
  basicstyle={\fontfamily{tx}\ttfamily\small},
  stringstyle=\color{green_ulises},
  commentstyle=\color{gray_ulisses},
  identifierstyle=\slshape{},
  keywordstyle=\bfseries,
  numberstyle=\small\color{gray_ulisses},
  backgroundcolor=\color{gray!5},
  numberblanklines=false,
  inputencoding={utf8},
  belowskip=-1mm,
  escapeinside={//*}{\^^M} % Allow to set labels and the like in comments
}

% Defines a custom environment for indented shell commands
\newenvironment{displayshellcommand}{%
	\begin{quote}%
	\ttfamily%
}{%
	\end{quote}%
}

% Define Idris2 language keywords
\lstdefinelanguage{Idris}{
    morekeywords={
        module, import, data, where, if, then, else, case, of, let, in, do, 
        Type, Int, Integer, Float, Double, Char, String, Bool, True, False, 
        List, Nil, (::), (->), (=>), Nat,(:),
    },
	morecomment=[l]{--},
}

\lstnewenvironment{haskell}{
  \vspace{1em}%
  \lstset{abovecaptionskip=1em, language=Haskell}
}{
  \vspace{1em}
}


\lstnewenvironment{idris}{
  \vspace{1em}%
  \lstset{
    language=idris,
  }
}{
  \vspace{1em}
}

% Define Agda listing style  
\lstdefinelanguage{Agda}{  
  keywords={  
    data, where, let, in, module, mutual, abstract, private,  
    public, postulate, primitive, record, field, constructor,  
    forall, Set, Type, with, rewrite, using, open, import,  
    instance, hiding, renaming, to  
  },  
  sensitive=true,  
  morecomment=[l]--,  
  morestring=[b]",  
  literate=  
    {∀}{{\ensuremath{\forall}}}1  
    {→}{{\ensuremath{\rightarrow}}}1  
    {←}{{\ensuremath{\leftarrow}}}1  
    {⊤}{{\ensuremath{\top}}}1  
    {⊥}{{\ensuremath{\bot}}}1  
    {∷}{{\ensuremath{::}}}1  
    {≡}{{\ensuremath{\equiv}}}1  
    {λ}{{\ensuremath{\lambda}}}1  
    {∘}{{\ensuremath{\circ}}}1  
    {∨}{{\ensuremath{\vee}}}1  
    {∧}{{\ensuremath{\wedge}}}1  
    {⊎}{{\ensuremath{\uplus}}}1  
    {×}{{\ensuremath{\times}}}1  
    {ℕ}{{\ensuremath{\mathbb{N}}}}1  
    {ℤ}{{\ensuremath{\mathbb{Z}}}}1  
    {⊢}{{\ensuremath{\vdash}}}1  
    {∈}{{\ensuremath{\in}}}1  
    {∉}{{\ensuremath{\notin}}}1  
    {∋}{{\ensuremath{\ni}}}1  
    {∌}{{\ensuremath{\not\ni}}}1  
}  

% Define the Agda environment  
\lstnewenvironment{agda}  
{\lstset{  
    language=Agda,  
}}{}  

\title{Idris2}
\event{Seminar: Programming Languages in Winter term 2024/2025}
\author{Mehrdad Shahidi, Zahra Khodabakhishan
  \institute{Rheinland-Pfälzische Technische Universität Kaiserslautern-Landau, Department of Computer Science}}
%%%%%%%%%%%%%%%%%%%%%%%%%%%%%%%%%%%%%%%%%%%%%%%%%%%%%%%%%%%%%%%%%%%%%%%%%%%%%%%
\begin{document}
%%%%%%%%%%%%%%%%%%%%%%%%%%%%%%%%%%%%%%%%%%%%%%%%%%%%%%%%%%%%%%%%%%%%%%%%%%%%%%%

\maketitle

%%%%%%%%%%%%%%%%%%%%%%%%%%%%%%%%%%%%%%%%%%%%%%%%%%%%%%%%%%%%%%%%%%%%%%%%%%%%%%%

\begin{abstract}
Idris2 is a functional programming language that leverages Quantitative Type Theory (QTT) to offer an advanced dependent type system, making it well-suited for both practical software development and formal proof verification.
This report provides example problem verification\todo{add problem name here} using Idris2 to showcase its capabilities, examines its QTT-based approach, compares its features with Agda, and concludes by summarizing key insights and suggesting directions for future exploration.

\end{abstract}

%%%%%%%%%%%%%%%%%%%%%%%%%%%%%%%%%%%%%%%%%%%%%%%%%%%%%%%%%%%%%%%%%%%%%%%%%%%%%%

\section{Introduction}
\label{sec:introduction}
As software systems become increasingly complex and critical to our daily operations, programming languages continue to evolve to meet growing demands for reliability and correctness.
Idris2 represents a significant advancement in this evolution by offering a unique approach to program correctness through its type system based on Quantitative Type Theory (QTT).
As a pure functional programming language with first-class types, Idris2 allows types to be manipulated and computed just like any other value, fundamentally shifting how we approach programming.
Unlike traditional programming languages that catch errors at runtime, Idris2 enables developers to work collaboratively with the compiler, which acts as an assistant during development, helping to prove properties about code and suggest potential solutions for incomplete programs.
This type-driven development approach ensures program correctness before execution, making it particularly valuable for critical systems, mathematical proofs, and complex state-dependent applications like concurrent systems\cite{BradyYoutube2023}.
\\example of idris2 code:
\begin{idris}
	safeDiv : (num : Nat) -> (d : Nat) -> {auto ok : GT d Z} -> Nat  
	safeDiv n d  = div n d  

	-- This works automatically because Idris can prove 0 < 5  
	example : Nat  
	example = safeDiv 12 5

	-- This code failes to compile because Idris can't prove 0 < 0
	example2 : Nat
	example2 = safeDiv 12 0


\end{idris}

The following sections provide an accessible overview of Idris2's key aspects.
We begin with a brief introduction to Quantitative Type Theory (QTT) and compare Idris2 with both Haskell and Agda, highlighting how these languages approach type systems differently.
While Haskell represents traditional functional programming and Agda focuses heavily on theorem proving, Idris2 aims to bridge the gap between practical programming and formal verification.
We then explore some distinctive features of Idris2 through simple examples, such as its interactive development environment and dependent types.
To demonstrate these concepts in action, we present a straightforward verification example that shows how Idris2's type system can prevent common programming errors.
The report concludes with a discussion of current limitations and practical considerations when using Idris2 in real-world scenarios.

\section{Background}  
\label{sec:background}  
Programming languages traditionally separate types and values, which means that certain kinds of errors, particularly those related to invalid values or incorrect data structures, are only caught at runtime. This limitation has been a significant challenge in software engineering, as it allows many types of errors—such as accessing an element from an empty list—to slip through until the program is executed. The goal of dependent types is to solve this problem by allowing types to depend on values, enabling the type system itself to enforce correctness at compile time.

The concept of dependent types is not new; it has roots in mathematical logic and type theory. However, its application in programming languages was limited for many years. Idris, first introduced by Edwin Brady in 2009, was a pioneering language that sought to make dependent types practical and usable in real-world programming. Idris was designed with the goal of enabling programmers to capture more precise specifications in the type system itself, making programs both more expressive and safer. Through its use of dependent types, Idris allowed for the formal verification of certain aspects of a program at compile time—such as proving that a vector is non-empty before calling the head function.

Despite its potential, Idris 1 had its limitations. The performance of type checking was not optimal for large programs, and the language’s implementation and theoretical foundations were not as robust as they could be. These shortcomings were particularly evident when trying to use Idris for large-scale or more complex applications. The need for improvement in both usability and performance led to the development of Idris 2.

Idris 2 represents a complete redesign of the language, drawing on the lessons learned from Idris 1. One of the key challenges in the original Idris was the efficiency of type checking, especially when dealing with large programs. Idris 2 addressed this by significantly improving its type checking performance. Furthermore, the language introduced the ability to erase compile-time arguments more effectively, reducing the overhead in generating executable code. Additionally, Idris 2 incorporated linear types through Quantitative Type Theory, providing a mechanism for better resource management and more precise control over side-effects in functional programming.

These improvements made Idris 2 a more practical and scalable tool for developers looking to leverage dependent types.
  

\section{Theorem Proving}
\label{sec:Propositions and judgments}
\subsection{Propositions and Judgments}
Before delving into theorem proving, it is essential to understand the foundational framework of constructive logic. Also known as intuitionistic logic, it differs from classical logic by rejecting the Law of Excluded Middle—which asserts that every proposition is either true or false. Instead, constructive logic considers a proposition true only if it can be explicitly proven. Thus, an unproven proposition is not necessarily false; it may simply be unprovable.

Constructive logic builds a database of judgments, where each judgment is a formally validated proof. This approach ensures the logical system’s integrity by requiring concrete evidence for every proposition. For instance, proving a number is even involves explicitly providing a number that satisfies the condition, rather than merely asserting it. By avoiding assumptions like the Law of Excluded Middle, constructive logic maintains consistency and reliability, ensuring that only provable propositions are included in the system.
\subsection{Equality}
\label{sec:Equality}
\subsubsection{definitional and propositional equality}
Equality in Idris is captured by the Equal type, defined as follows:
\begin{idris}
data (=) : a -> b -> Type where  
    Reflexive : x = x
\end{idris}
This states that two values are equal if they are definitionally identical, with Reflexive serving as explicit evidence of their equality.

\begin{idris}
    plusReducesL : (n : Nat) -> plus Z n = n
    plusReducesL n = Reflexive
\end{idris}
This proposition asserts that adding zero to any number n results in n, which is definitionally true.
\begin{idris}
    plusReducesR : (n : Nat) -> plus n Z = n
    plusReducesR n = Reflexive
\end{idris}
While plusReducesL is definitionally equal and proved by Reflexive, plusReducesR is not, because the function plus is defined by recursion on its first argument. When the first argument is Z, it reduces, but not when the second argument is Z. Therefore, Reflexive cannot prove plusReducesR directly.
If an equality type can be proved using Reflexive alone, it is known as a definitional equality.
\subsubsection{Heterogeneous Equality}
Equality in Idris is heterogeneous, meaning that it allows for the possibility of proposing equalities between values of different types. It means suggesting that two values of different types can be considered equal under certain assumptions or relationships. It becomes essential when working with dependent types, where types themselves depend on values.
\begin{idris}
    vect_eq_length : (xs : Vect n a) -> (ys : Vect m a) ->
    (xs = ys) -> n = m 
\end{idris}
(xs = ys) asserts that the two vectors are equal in both value and type.So if xs = ys, then n and m must also be equal, because the length of the vector (n or m) is part of its type. This is why the function can safely return n = m.
\section{Features of Idris2}  
\label{sec:features}
\subsection{Multiplicities}
Quantitative Type Theory (QTT) extends traditional dependent type theory by adding explicit tracking of how variables are used in programs \cite{atkey2018syntax}. 

In QTT, each variable binding (including function arguments)
has a multiplicity. QTT itself takes a general approach to multiplicities, allowing any semiring.
For Idris 2, we make a concrete choice of semiring, where a multiplicity can be one of:
\begin{lstlisting}
    0: The variable is used only at compile time and erased at runtime
    1: The variable must be used exactly once at runtime (linear)
    omega: The variable can be used any number of times (unrestricted)
\end{lstlisting}

Multiplicities in Idris 2 describe how often a variable must be used within the scope of its binding. A variable is considered "used" when it appears in the body of a definition, as opposed to a type declaration, and when it is passed as an argument with multiplicity 1 or \(\omega\). The multiplicities of a function's arguments are specified by its type, and they dictate how many times the arguments can be used within the function's body. Variables with multiplicity \(\omega\) are considered unrestricted: they can be passed to argument positions with multiplicities 0, 1, or \(\omega\). On the other hand, a function that accepts an argument with multiplicity 1 guarantees that the argument will not be shared within its body in the future, although it is not required to ensure that it has not been shared in the past\cite{brady2021idris}.
\subsection{Interface}
we use interfaces (similar to type classes in Haskell) to define functions or operations that can work seamlessly across different types. This capability, known as overloading, is achieved by defining a common interface (a set of operations) and then providing specific implementations for each type that becomes an instance of the interface.

\subsubsection{Implementing Show for Nat :}
the Show interface is used to convert values into String.The key rule is uniqueness of the implementation , There should be only one implementation of am interface for specific type.
\begin{idris}
    Show Nat where
    show Z     = "Zero"
    show (S Z) = "One"
    show (S (S k)) = "Two"
    show (S (S (S k))) = "S(" ++ show (S (S k)) ++ ")"
  
\end{idris}
Additionally, Implementations can have constraints. For instance, to implement Show for a vector (Vect n a), there must already be a Show implementation for the element type a.
\begin{idris}
    Show a => Show (Vect n a) where
\end{idris}
This is a constraint. It means the implementation of Show for Vect n a (a vector of n elements of type a) is only valid if the element type a itself has an instance of Show. This is a constraint. It means the implementation of Show for Vect n a (a vector of n elements of type a) is only valid if the element type a itself has an instance of Show.
\begin{idris}
    Show a => Show (Vect n a) where
    show xs = "[" ++ show' xs ++ "]" where
        show' : forall n . Vect n a -> String
        show' Nil        = ""
        show' (x :: Nil) = show x
        show' (x :: xs)  = show x ++ ", " ++ show' xs
\end{idris}
\subsubsection{Type classes in Idris vs Haskell} 
Type classes are a central feature of both Haskell and Idris 2. While they share many similarities, Idris 2's more expressive dependent type system enables additional flexibility and power.

In Idris 2, type classes leverage dependent types, allowing types to depend on values. This extends polymorphism beyond types (as in Haskell) to include relationships between types and values, enabling more expressive and precise constraints. Haskell lacks this capability due to its non-dependent type system.

In this Haskell example, we'll define a type class Summable for containers that can hold numeric values. The sumElements function will sum the elements of a container, but it will only work for the container's type, not based on the size of the container.
\begin{haskell}
  class Summable c where
  sumElements :: Num a => c a -> a

instance Summable [] where
  sumElements xs = sum xs  
\end{haskell}
In Haskell, type classes like Summable work based on the type of the container ([] in our case) but have no knowledge of the container's size(Type-Only Polymorphism). The behavior is generalized and works for any size of container without any specialized handling.
\begin{idris}
    interface Summable (n : Nat) (c : Type) where
    sumElements : Vect n c -> Int  
  
  
    implementation Summable n Int where
    sumElements [] = 0  
    sumElements (x :: xs) = x + sumElements xs  
  
\end{idris}
we can define the container's size as part of the type, enabling value-dependent polymorphism. This means we can write type classes and functions that depend on both the type and size of the container. For example, the sum of a vector is defined based on the number of elements in the vector.
\subsection{Interactive Development Environment}  
\label{subsec:ide}  
% Type-driven development workflow  
% Hole mechanism  
% Interactive editing features  

\subsection{Dependent Types in Practice}  
\label{subsec:dependent-types}  
% Real-world applications  
% Code examples  
% Common patterns and idioms  

\section{Verification Example}  
\label{sec:verification-example}  
\subsection{Problem Description}  
\label{subsec:problem-desc}  
% Detailed description of the chosen verification problem  
% Motivation and relevance  

\subsection{Implementation and Proof}  
\label{subsec:implementation}  
% Step-by-step solution  
% Code with explanations  
% Proof strategy  

\section{Practical Considerations}  
\label{sec:practical-considerations}  
\subsection{Current Limitations}  
\label{subsec:limitations}  
% Technical limitations  
% Ecosystem challenges  
% Performance considerations  

\subsection{Future Directions}  
\label{subsec:future}  
% Potential improvements  
% Research opportunities  
% Community development  

\section{Conclusion}  
\label{sec:conclusion}  
% Summary of key points  
% Final thoughts on Idris2's role in programming  
% Recommendations for adoption  

\newpage
\nocite{*}
\bibliographystyle{eptcs}
\bibliography{references}

%%%%%%%%%%%%%%%%%%%%%%%%%%%%%%%%%%%%%%%%%%%%%%%%%%%%%%%%%%%%%%%%%%%%%%%%%%%%%%%
\end{document}
%%%%%%%%%%%%%%%%%%%%%%%%%%%%%%%%%%%%%%%%%%%%%%%%%%%%%%%%%%%%%%%%%%%%%%%%%%%%%%%
