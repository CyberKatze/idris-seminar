\documentclass[]{rptuseminar}

% Specify that the source file has UTF8 encoding
\usepackage[utf8]{inputenc}
% Set up the document font; font encoding (here T1) has to fit the used font.
\usepackage[T1]{fontenc}
\usepackage{lmodern}

% Load language spec
\usepackage[american]{babel}
% German article --> ngerman (n for »neue deutsche Rechtschreibung«)
% British English --> english

% Ffor bibliography and \cite
\usepackage{cite}

% AMS extensions for math typesetting
\usepackage[intlimits]{mathtools}
\usepackage{amssymb}
% ... there are many more ...


% Load \todo command for notes
\usepackage{todonotes}
% Sebastian's favorite command for large inline todonotes
% Caveat: does not work well with \listoftodos
\newcommand\todoin[2][]{\todo[inline, caption={2do}, #1]{
		\begin{minipage}{\linewidth-1em}\noindent\relax#2\end{minipage}}}

% Load \includegraphics command for including pictures (pdf or png highly recommended)
\usepackage{graphicx}

% Typeset source/pseudo code
\usepackage{listings}

% Load TikZ library for creating graphics
% Using the PGF/TikZ manual and/or tex.stackexchange.com is highly adviced.
\usepackage{tikz}
% Load tikz libraries needed below (see the manual for a full list)
\usetikzlibrary{automata,positioning}

% Load \url command for easier hyperlinks without special link text
\usepackage{url}

% Load support for links in pdfs
\usepackage{hyperref}
\setlength{\footnotesep}{10pt}

% Defines default styling for code listings
\definecolor{gray_ulisses}{gray}{0.55}
\definecolor{green_ulises}{rgb}{0.2,0.75,0}
\lstset{%
  columns=flexible,
  keepspaces=true,
  tabsize=2,
  basicstyle={\fontfamily{tx}\ttfamily\small},
  stringstyle=\color{green_ulises},
  commentstyle=\color{gray_ulisses},
  identifierstyle=\slshape{},
  keywordstyle=\bfseries,
  numberstyle=\small\color{gray_ulisses},
  backgroundcolor=\color{gray!5},
  numberblanklines=false,
  inputencoding={utf8},
  belowskip=-1mm,
  escapeinside={//*}{\^^M} % Allow to set labels and the like in comments
}

% Defines a custom environment for indented shell commands
\newenvironment{displayshellcommand}{%
	\begin{quote}%
	\ttfamily%
}{%
	\end{quote}%
}

% Define Idris2 language keywords
\lstdefinelanguage{Idris}{
    morekeywords={
        module, import, data, where, if, then, else, case, of, let, in, do, 
        Type, Int, Integer, Float, Double, Char, String, Bool, True, False, 
        List, Nil, (::), (->), (=>), Nat,(:),
    },
	morecomment=[l]{--},
}

\lstnewenvironment{haskell}{
  \vspace{1em}%
  \lstset{abovecaptionskip=1em, language=Haskell}
}{
  \vspace{1em}
}


\lstnewenvironment{idris}{
  \vspace{1em}%
  \lstset{
    language=idris,
  }
}{
  \vspace{1em}
}

% Define Agda listing style  
\lstdefinelanguage{Agda}{  
  keywords={  
    data, where, let, in, module, mutual, abstract, private,  
    public, postulate, primitive, record, field, constructor,  
    forall, Set, Type, with, rewrite, using, open, import,  
    instance, hiding, renaming, to  
  },  
  sensitive=true,  
  morecomment=[l]--,  
  morestring=[b]",  
  literate=  
    {∀}{{\ensuremath{\forall}}}1  
    {→}{{\ensuremath{\rightarrow}}}1  
    {←}{{\ensuremath{\leftarrow}}}1  
    {⊤}{{\ensuremath{\top}}}1  
    {⊥}{{\ensuremath{\bot}}}1  
    {∷}{{\ensuremath{::}}}1  
    {≡}{{\ensuremath{\equiv}}}1  
    {λ}{{\ensuremath{\lambda}}}1  
    {∘}{{\ensuremath{\circ}}}1  
    {∨}{{\ensuremath{\vee}}}1  
    {∧}{{\ensuremath{\wedge}}}1  
    {⊎}{{\ensuremath{\uplus}}}1  
    {×}{{\ensuremath{\times}}}1  
    {ℕ}{{\ensuremath{\mathbb{N}}}}1  
    {ℤ}{{\ensuremath{\mathbb{Z}}}}1  
    {⊢}{{\ensuremath{\vdash}}}1  
    {∈}{{\ensuremath{\in}}}1  
    {∉}{{\ensuremath{\notin}}}1  
    {∋}{{\ensuremath{\ni}}}1  
    {∌}{{\ensuremath{\not\ni}}}1  
}  

% Define the Agda environment  
\lstnewenvironment{agda}  
{\lstset{  
    language=Agda,  
}}{}  

\title{Idris2}
\event{Seminar: Programming Languages in Winter term 2024/2025}
\author{Mehrdad Shahidi, Zahra Khodabakhishan
  \institute{Rheinland-Pfälzische Technische Universität Kaiserslautern-Landau, Department of Computer Science}}
%%%%%%%%%%%%%%%%%%%%%%%%%%%%%%%%%%%%%%%%%%%%%%%%%%%%%%%%%%%%%%%%%%%%%%%%%%%%%%%
\begin{document}
%%%%%%%%%%%%%%%%%%%%%%%%%%%%%%%%%%%%%%%%%%%%%%%%%%%%%%%%%%%%%%%%%%%%%%%%%%%%%%%

\maketitle

%%%%%%%%%%%%%%%%%%%%%%%%%%%%%%%%%%%%%%%%%%%%%%%%%%%%%%%%%%%%%%%%%%%%%%%%%%%%%%%

\begin{abstract}
Idris2 is a functional programming language that leverages Quantitative Type Theory (QTT) to offer an advanced dependent type system, making it well-suited for both practical software development and formal proof verification.
This report provides example problem verification\todo{add problem name here} using Idris2 to showcase its capabilities, examines its QTT-based approach, compares its features with Agda, and concludes by summarizing key insights and suggesting directions for future exploration.

\end{abstract}

%%%%%%%%%%%%%%%%%%%%%%%%%%%%%%%%%%%%%%%%%%%%%%%%%%%%%%%%%%%%%%%%%%%%%%%%%%%%%%

\section{Introduction}
\label{sec:introduction}
As software systems become increasingly complex and critical to our daily operations, programming languages continue to evolve to meet growing demands for reliability and correctness.
Idris2 represents a significant advancement in this evolution by offering a unique approach to program correctness through its type system based on Quantitative Type Theory (QTT).
As a pure functional programming language with first-class types, Idris2 allows types to be manipulated and computed just like any other value, fundamentally shifting how we approach programming.
Unlike traditional programming languages that catch errors at runtime, Idris2 enables developers to work collaboratively with the compiler, which acts as an assistant during development, helping to prove properties about code and suggest potential solutions for incomplete programs.
This type-driven development approach ensures program correctness before execution, making it particularly valuable for critical systems, mathematical proofs, and complex state-dependent applications like concurrent systems\cite{BradyYoutube2023}.
\\example of idris2 code:
\begin{idris}
	safeDiv : (num : Nat) -> (d : Nat) -> {auto ok : GT d Z} -> Nat  
	safeDiv n d  = div n d  

	-- This works automatically because Idris can prove 0 < 5  
	example : Nat  
	example = safeDiv 12 5

	-- This code failes to compile because Idris can't prove 0 < 0
	example2 : Nat
	example2 = safeDiv 12 0


\end{idris}

The following sections provide an accessible overview of Idris2's key aspects.
We begin with a brief introduction to Quantitative Type Theory (QTT) and compare Idris2 with both Haskell and Agda, highlighting how these languages approach type systems differently.
While Haskell represents traditional functional programming and Agda focuses heavily on theorem proving, Idris2 aims to bridge the gap between practical programming and formal verification.
We then explore some distinctive features of Idris2 through simple examples, such as its interactive development environment and dependent types.
To demonstrate these concepts in action, we present a straightforward verification example that shows how Idris2's type system can prevent common programming errors.
The report concludes with a discussion of current limitations and practical considerations when using Idris2 in real-world scenarios.

\section{Background}  
\label{sec:background}  
Idris, first introduced by Edwin Brady in 2009, emerged as a programming language designed to explore dependent types in a practical programming context.  
The language drew inspiration from both Haskell's practical functional programming approach and Agda's powerful type system.  
While the original Idris (now known as Idris1) successfully demonstrated the potential of dependent types in practical programming, it also revealed certain limitations in its implementation and theoretical foundations \cite{brady2013idris}.  

These insights led to the development of Idris2, a complete redesign of the language implemented in Idris1 itself.  
The new version introduced several significant improvements, including:  
\begin{itemize}  
    \item Better type checking performance  
    \item Improved erasure of compile-time-only arguments  
    \item Linear types through Quantitative Type Theory  
    \item A more robust implementation based on experience with Idris1  
\end{itemize}  

A key example demonstrating Idris2's practical application of dependent types is the classic vector length safety:  
\begin{idris}  
data Vect : Nat -> Type -> Type where  
    Nil  : Vect Z a  
    (::) : a -> Vect k a -> Vect (S k) a  

-- Safe head function that can only be called on non-empty vectors  
head : Vect (S n) a -> a  
head (x :: _) = x  

-- This compiles  
example1 : Integer  
example1 = head [1,2,3]  

-- This fails to compile as the vector might be empty  
example2 : Maybe Integer  
example2 = head []  
\end{idris}  
\subsection{Quantitative Type Theory Foundations}  
\label{subsec:qtt-foundations}  
Quantitative Type Theory (QTT) extends traditional dependent type theory by adding explicit tracking of how variables are used in programs \cite{atkey2018syntax}. In Idris2, QTT serves two main purposes:  
\begin{itemize}  
    \item Providing clear type-level guarantees about which values are required at runtime  
    \item Enabling precise tracking of resource usage through linear types  
\end{itemize}  

\subsubsection{Core Concepts}  
QTT in Idris2 uses three fundamental quantities that specify how variables can be used:  
\begin{itemize}  
    \item 0: The variable is used only at compile time and erased at runtime  
    \item 1: The variable must be used exactly once at runtime (linear)  
    \item $\omega$: The variable can be used any number of times (unrestricted)  
\end{itemize}  

For example, here's how quantities work in practice:  
\begin{idris}  
-- The 0 quantity means 'n' is erased at runtime  
length : {0 n : Nat} -> Vect n a -> Nat  
length [] = 0  
length (x :: xs) = 1 + length xs  

-- The 1 quantity ensures the file handle is used exactly once  
writeAndClose : (1 handle : File) -> IO ()  
writeAndClose handle = do  
    writeFile handle "Hello"  
    closeFile handle  -- Must close the file exactly once  
\end{idris}  

\subsubsection{Benefits of QTT}  
The integration of QTT in Idris2 provides several key advantages:  
\begin{itemize}  
    \item \textbf{Erasure Control}: Clear type-level specifications of which values are needed at runtime  
    \item \textbf{Resource Safety}: Linear types ensure resources are used exactly once  
    \item \textbf{Performance}: Better control over runtime behavior through erasure  
    \item \textbf{Protocol Safety}: Ability to enforce correct usage of protocols and state machines  
\end{itemize}  

\section{Comparing Idris2 with Haskell and Agda}  
While Idris2, Haskell, and Agda are all functional programming languages with strong type systems, they each serve different purposes and offer distinct approaches to type theory and practical programming. This section examines their key differences and similarities across various aspects.  

\subsection{Type System Characteristics}  
\begin{itemize}  
    \item \textbf{Haskell}: Features a Hindley-Milner type system\footnote{Named after Roger Hindley and Robin Milner for their foundational work on type inference \cite{hindley1969principal,milner1978theory}.} with type classes, offering strong static typing without dependent types. Type-level programming is possible through extensions but is not a core feature.  
    
    \item \textbf{Agda}: Implements a dependent type system based on Martin-Löf Type Theory\footnote{Martin-Löf's type theory \cite{martin1984intuitionistic} introduced dependent types and laid the foundation for modern proof assistants like Agda.}, primarily focused on theorem proving and formal verification. Its type system is more expressive than Haskell's but requires more manual proof work.  
    
    \item \textbf{Idris2}: Combines practical programming with dependent types through QTT, offering a middle ground between Haskell's practicality and Agda's theorem-proving capabilities. Its type system supports both runtime and compile-time type checking with explicit erasure control.  
\end{itemize}  

\subsection{Practical Programming Aspects}  
Consider the following example implementing a safe list indexing function in all three languages:  

\begin{haskell}
-- Haskell: Runtime safety through Maybe type  
(!?) :: [a] -> Int -> Maybe a  
xs !? n | n < 0     = Nothing  
        | otherwise = case drop n xs of  
                       []     -> Nothing  
                       (x:_)  -> Just x  
\end{haskell}

\begin{agda}
-- Agda: Compile-time safety through dependent types  
_!!_ : ∀ {A : Set} {n} → Vec A n → Fin n → A  
(x ∷ xs) !! zero  = x  
(x ∷ xs) !! suc i = xs !! i  
\end{agda}

\begin{idris}  
-- Idris2: Practical dependent types with erasure  
index : (i : Fin n) -> Vect n a -> a  
index FZ     (x :: xs) = x  
index (FS k) (x :: xs) = index k xs  
\end{idris}  

Key differences in practical usage include:  
\begin{itemize}  
    \item \textbf{Development Experience}:  
        \begin{itemize}  
            \item Haskell offers the most mature ecosystem and tooling  
            \item Agda focuses on interactive theorem proving  
            \item Idris2 provides interactive type-driven development with practical programming features  
        \end{itemize}  
    
    \item \textbf{Performance Considerations}:  
        \begin{itemize}  
            \item Haskell generally offers the best runtime performance  
            \item Agda's focus is on correctness rather than performance  
            \item Idris2's QTT enables better performance through erasure of compile-time-only values  
        \end{itemize}  
\end{itemize}  

\subsection{Theorem Proving Capabilities}  
The languages differ significantly in their approach to formal verification:  

\begin{itemize}  
    \item \textbf{Haskell}:   
        \begin{itemize}  
            \item Limited built-in theorem proving capabilities  
            \item Relies on external tools like LiquidHaskell for formal verification  
            \item Properties often verified through testing rather than proofs  
        \end{itemize}  
    
    \item \textbf{Agda}:  
        \begin{itemize}  
            \item First-class support for theorem proving  
            \item Extensive library for mathematical proofs  
            \item Requires explicit proof terms  
        \end{itemize}  
    
    \item \textbf{Idris2}:  
        \begin{itemize}  
            \item Balanced approach to theorem proving  
            \item Automated proof search capabilities  
            \item Integration of proofs with practical programming  
        \end{itemize}  
\end{itemize}  

\subsection{Syntax and Learning Curve}  
The syntax and learning curve vary significantly:  

\begin{itemize}  
    \item \textbf{Haskell}: Most accessible syntax, familiar to many functional programmers  
    \item \textbf{Agda}: Steeper learning curve, requires understanding of type theory  
    \item \textbf{Idris2}: Haskell-like syntax with additional complexity from dependent types  
\end{itemize}  

An example showing similar functions across the three languages:  

\begin{haskell}  
-- Haskell: Simple pattern matching  
reverse :: [a] -> [a]  
reverse [] = []  
reverse (x:xs) = reverse xs ++ [x]  
\end{haskell}  

\begin{agda}
-- Agda: Explicit type universe and termination checking  
reverse : ∀ {A : Set} → List A → List A  
reverse []       = []  
reverse (x ∷ xs) = reverse xs ++ (x ∷ [])  
\end{agda}
\begin{idris}
-- Idris2: Dependent types with erasure  
reverse : {0 a : Type} -> List a -> List a  
reverse [] = []  
reverse (x :: xs) = reverse xs ++ [x]  
\end{idris}  

\subsection{Use Case Recommendations}  
Based on these comparisons, each language is best suited for different scenarios:  

\begin{itemize}  
    \item \textbf{Choose Haskell} for:  
        \begin{itemize}  
            \item Production-ready software development  
            \item Large-scale applications  
            \item When a mature ecosystem is crucial  
        \end{itemize}  
    
    \item \textbf{Choose Agda} for:  
        \begin{itemize}  
            \item Mathematical proofs and formalization  
            \item Academic research in type theory  
            \item When maximum proof power is needed  
        \end{itemize}  
    
    \item \textbf{Choose Idris2} for:  
        \begin{itemize}  
            \item Combining practical programming with formal verification  
            \item Projects requiring precise resource management  
            \item When dependent types need to coexist with practical features  
        \end{itemize}  
\end{itemize}
\section{Key Features of Idris2}  
\label{sec:key-features}  
\subsection{Interactive Development Environment}  
\label{subsec:ide}  
% Type-driven development workflow  
% Hole mechanism  
% Interactive editing features  

\subsection{Dependent Types in Practice}  
\label{subsec:dependent-types}  
% Real-world applications  
% Code examples  
% Common patterns and idioms  

\section{Verification Example}  
\label{sec:verification-example}  
\subsection{Problem Description}  
\label{subsec:problem-desc}  
% Detailed description of the chosen verification problem  
% Motivation and relevance  

\subsection{Implementation and Proof}  
\label{subsec:implementation}  
% Step-by-step solution  
% Code with explanations  
% Proof strategy  

\section{Practical Considerations}  
\label{sec:practical-considerations}  
\subsection{Current Limitations}  
\label{subsec:limitations}  
% Technical limitations  
% Ecosystem challenges  
% Performance considerations  

\subsection{Future Directions}  
\label{subsec:future}  
% Potential improvements  
% Research opportunities  
% Community development  

\section{Conclusion}  
\label{sec:conclusion}  
% Summary of key points  
% Final thoughts on Idris2's role in programming  
% Recommendations for adoption  

\newpage
\nocite{*}
\bibliographystyle{eptcs}
\bibliography{references}

%%%%%%%%%%%%%%%%%%%%%%%%%%%%%%%%%%%%%%%%%%%%%%%%%%%%%%%%%%%%%%%%%%%%%%%%%%%%%%%
\end{document}
%%%%%%%%%%%%%%%%%%%%%%%%%%%%%%%%%%%%%%%%%%%%%%%%%%%%%%%%%%%%%%%%%%%%%%%%%%%%%%%
