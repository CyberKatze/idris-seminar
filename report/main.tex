\documentclass[]{rptuseminar}

% Specify that the source file has UTF8 encoding
\usepackage[utf8]{inputenc}
% Set up the document font; font encoding (here T1) has to fit the used font.
\usepackage[T1]{fontenc}
\usepackage{lmodern}

% Load language spec
\usepackage[american]{babel}
% German article --> ngerman (n for »neue deutsche Rechtschreibung«)
% British English --> english

% Ffor bibliography and \cite
\usepackage{cite}

% AMS extensions for math typesetting
\usepackage[intlimits]{mathtools}
\usepackage{amssymb}
% ... there are many more ...


% Load \todo command for notes
\usepackage{todonotes}
% Sebastian's favorite command for large inline todonotes
% Caveat: does not work well with \listoftodos
\newcommand\todoin[2][]{\todo[inline, caption={2do}, #1]{
		\begin{minipage}{\linewidth-1em}\noindent\relax#2\end{minipage}}}

% Load \includegraphics command for including pictures (pdf or png highly recommended)
\usepackage{graphicx}

% Typeset source/pseudo code
\usepackage{listings}

% Load TikZ library for creating graphics
% Using the PGF/TikZ manual and/or tex.stackexchange.com is highly adviced.
\usepackage{tikz}
% Load tikz libraries needed below (see the manual for a full list)
\usetikzlibrary{automata,positioning}

% Load \url command for easier hyperlinks without special link text
\usepackage{url}

% Load support for links in pdfs
\usepackage{hyperref}

% Defines default styling for code listings
\definecolor{gray_ulisses}{gray}{0.55}
\definecolor{green_ulises}{rgb}{0.2,0.75,0}
\lstset{%
  columns=flexible,
  keepspaces=true,
  tabsize=3,
  basicstyle={\fontfamily{tx}\ttfamily\small},
  stringstyle=\color{green_ulises},
  commentstyle=\color{gray_ulisses},
  identifierstyle=\slshape{},
  keywordstyle=\bfseries,
  numberstyle=\small\color{gray_ulisses},
  numberblanklines=false,
  inputencoding={utf8},
  belowskip=-1mm,
  escapeinside={//*}{\^^M} % Allow to set labels and the like in comments
}

% Defines a custom environment for indented shell commands
\newenvironment{displayshellcommand}{%
	\begin{quote}%
	\ttfamily%
}{%
	\end{quote}%
}

% Define Idris2 language keywords
\lstdefinelanguage{Idris}{
    morekeywords={
        module, import, data, where, if, then, else, case, of, let, in, do, 
        Type, Int, Integer, Float, Double, Char, String, Bool, True, False, 
        List, Nil, (::), (->), (=>), Nat,(:),
    },
	morecomment=[l]{--},
}

\lstnewenvironment{idris}{
  \vspace{1em}%
  \lstset{
    language=idris,
    columns=flexible,
    keepspaces=true,
    tabsize=2,
    basicstyle={\fontfamily{tx}\ttfamily\small},
    stringstyle=\color{green_ulises},
    commentstyle=\color{gray_ulisses},
    identifierstyle=\slshape{},
    keywordstyle=\bfseries,
    numberstyle=\small\color{gray_ulisses},
    backgroundcolor=\color{gray!5},
    numberblanklines=false,
    inputencoding={utf8},
    belowskip=-1mm,
    escapeinside={//*}{\^^M} % Allow to set labels and the like in comments
  }
}{
  \vspace{1em}
}


\title{Idris2}
\event{Seminar: Programming Languages in Winter term 2024/2025}
\author{Mehrdad Shahidi, Zahra Khodabakhishan
  \institute{Rheinland-Pfälzische Technische Universität Kaiserslautern-Landau, Department of Computer Science}}
%%%%%%%%%%%%%%%%%%%%%%%%%%%%%%%%%%%%%%%%%%%%%%%%%%%%%%%%%%%%%%%%%%%%%%%%%%%%%%%
\begin{document}
%%%%%%%%%%%%%%%%%%%%%%%%%%%%%%%%%%%%%%%%%%%%%%%%%%%%%%%%%%%%%%%%%%%%%%%%%%%%%%%

\maketitle

%%%%%%%%%%%%%%%%%%%%%%%%%%%%%%%%%%%%%%%%%%%%%%%%%%%%%%%%%%%%%%%%%%%%%%%%%%%%%%%

\begin{abstract}
Idris2 is a functional programming language that leverages Quantitative Type Theory (QTT) to offer an advanced dependent type system, making it well-suited for both practical software development and formal proof verification.
This report provides example problem verification\todo{add problem name here} using Idris2 to showcase its capabilities, examines its QTT-based approach, compares its features with Agda, and concludes by summarizing key insights and suggesting directions for future exploration.

\end{abstract}

%%%%%%%%%%%%%%%%%%%%%%%%%%%%%%%%%%%%%%%%%%%%%%%%%%%%%%%%%%%%%%%%%%%%%%%%%%%%%%

\section{Introduction}
\label{sec:introduction}
As software systems become increasingly complex and critical to our daily operations, programming languages continue to evolve to meet growing demands for reliability and correctness.
Idris2 represents a significant advancement in this evolution by offering a unique approach to program correctness through its type system based on Quantitative Type Theory (QTT).
As a pure functional programming language with first-class types, Idris2 allows types to be manipulated and computed just like any other value, fundamentally shifting how we approach programming.
Unlike traditional programming languages that catch errors at runtime, Idris2 enables developers to work collaboratively with the compiler, which acts as an assistant during development, helping to prove properties about code and suggest potential solutions for incomplete programs.
This type-driven development approach ensures program correctness before execution, making it particularly valuable for critical systems, mathematical proofs, and complex state-dependent applications like concurrent systems\cite{BradyYoutube2023}.
\\example of idris2 code:
\begin{idris}
	safeDiv : (num : Nat) -> (d : Nat) -> {auto ok : GT d Z} -> Nat  
	safeDiv n d  = div n d  

	-- This works automatically because Idris can prove 0 < 5  
	example : Nat  
	example = safeDiv 12 5

	-- This code failes to compile because Idris can't prove 0 < 0
	example2 : Nat
	example2 = safeDiv 12 0


\end{idris}

The following sections provide an accessible overview of Idris2's key aspects.
We begin with a brief introduction to Quantitative Type Theory (QTT) and compare Idris2 with both Haskell and Agda, highlighting how these languages approach type systems differently.
While Haskell represents traditional functional programming and Agda focuses heavily on theorem proving, Idris2 aims to bridge the gap between practical programming and formal verification.
We then explore some distinctive features of Idris2 through simple examples, such as its interactive development environment and dependent types.
To demonstrate these concepts in action, we present a straightforward verification example that shows how Idris2's type system can prevent common programming errors.
The report concludes with a discussion of current limitations and practical considerations when using Idris2 in real-world scenarios.

\section{Background}  
\label{sec:background}  
Idris, first introduced by Edwin Brady in 2009, emerged as a programming language designed to explore dependent types in a practical programming context.  
The language drew inspiration from both Haskell's practical functional programming approach and Agda's powerful type system.  
While the original Idris (now known as Idris1) successfully demonstrated the potential of dependent types in practical programming, it also revealed certain limitations in its implementation and theoretical foundations \cite{brady2013idris}.  

These insights led to the development of Idris2, a complete redesign of the language implemented in Idris1 itself.  
The new version introduced several significant improvements, including:  
\begin{itemize}  
    \item Better type checking performance  
    \item Improved erasure of compile-time-only arguments  
    \item Linear types through Quantitative Type Theory  
    \item A more robust implementation based on experience with Idris1  
\end{itemize}  

A key example demonstrating Idris2's practical application of dependent types is the classic vector length safety:  
\begin{idris}  
data Vect : Nat -> Type -> Type where  
    Nil  : Vect Z a  
    (::) : a -> Vect k a -> Vect (S k) a  

-- Safe head function that can only be called on non-empty vectors  
head : Vect (S n) a -> a  
head (x :: _) = x  

-- This compiles  
example1 : Integer  
example1 = head [1,2,3]  

-- This fails to compile as the vector might be empty  
example2 : Maybe Integer  
example2 = head []  
\end{idris}  
\subsection{Quantitative Type Theory Foundations}  
\label{subsec:qtt-foundations}  
Quantitative Type Theory (QTT) extends traditional dependent type theory by adding explicit tracking of how variables are used in programs \cite{atkey2018syntax}. In Idris2, QTT serves two main purposes:  
\begin{itemize}  
    \item Providing clear type-level guarantees about which values are required at runtime  
    \item Enabling precise tracking of resource usage through linear types  
\end{itemize}  

\subsubsection{Core Concepts}  
QTT in Idris2 uses three fundamental quantities that specify how variables can be used:  
\begin{itemize}  
    \item 0: The variable is used only at compile time and erased at runtime  
    \item 1: The variable must be used exactly once at runtime (linear)  
    \item $\omega$: The variable can be used any number of times (unrestricted)  
\end{itemize}  

For example, here's how quantities work in practice:  
\begin{idris}  
-- The 0 quantity means 'n' is erased at runtime  
length : {0 n : Nat} -> Vect n a -> Nat  
length [] = 0  
length (x :: xs) = 1 + length xs  

-- The 1 quantity ensures the file handle is used exactly once  
writeAndClose : (1 handle : File) -> IO ()  
writeAndClose handle = do  
    writeFile handle "Hello"  
    closeFile handle  -- Must close the file exactly once  
\end{idris}  

\subsubsection{Benefits of QTT}  
The integration of QTT in Idris2 provides several key advantages:  
\begin{itemize}  
    \item \textbf{Erasure Control}: Clear type-level specifications of which values are needed at runtime  
    \item \textbf{Resource Safety}: Linear types ensure resources are used exactly once  
    \item \textbf{Performance}: Better control over runtime behavior through erasure  
    \item \textbf{Protocol Safety}: Ability to enforce correct usage of protocols and state machines  
\end{itemize}  

\section{Comparing Idris2 with Haskell and Agda}  
% Type system differences  
% Practical programming aspects  
% Code examples highlighting key differences  
% Theorem proving capabilities  
% Syntax and usability comparison  
% Different approaches to dependent types  

\section{Key Features of Idris2}  
\label{sec:key-features}  
\subsection{Interactive Development Environment}  
\label{subsec:ide}  
% Type-driven development workflow  
% Hole mechanism  
% Interactive editing features  

\subsection{Dependent Types in Practice}  
\label{subsec:dependent-types}  
% Real-world applications  
% Code examples  
% Common patterns and idioms  

\section{Verification Example}  
\label{sec:verification-example}  
\subsection{Problem Description}  
\label{subsec:problem-desc}  
% Detailed description of the chosen verification problem  
% Motivation and relevance  

\subsection{Implementation and Proof}  
\label{subsec:implementation}  
% Step-by-step solution  
% Code with explanations  
% Proof strategy  

\section{Practical Considerations}  
\label{sec:practical-considerations}  
\subsection{Current Limitations}  
\label{subsec:limitations}  
% Technical limitations  
% Ecosystem challenges  
% Performance considerations  

\subsection{Future Directions}  
\label{subsec:future}  
% Potential improvements  
% Research opportunities  
% Community development  

\section{Conclusion}  
\label{sec:conclusion}  
% Summary of key points  
% Final thoughts on Idris2's role in programming  
% Recommendations for adoption  

\newpage
\nocite{*}
\bibliographystyle{eptcs}
\bibliography{references}

%%%%%%%%%%%%%%%%%%%%%%%%%%%%%%%%%%%%%%%%%%%%%%%%%%%%%%%%%%%%%%%%%%%%%%%%%%%%%%%
\end{document}
%%%%%%%%%%%%%%%%%%%%%%%%%%%%%%%%%%%%%%%%%%%%%%%%%%%%%%%%%%%%%%%%%%%%%%%%%%%%%%%
